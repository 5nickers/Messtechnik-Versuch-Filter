\documentclass[11pt]{scrartcl}
\usepackage[T1]{fontenc}
\usepackage[a4paper, left=3cm, right=2cm, top=2cm, bottom=2cm]{geometry}
\usepackage[activate]{pdfcprot}
\usepackage[ngerman]{babel}
\usepackage[parfill]{parskip}
\usepackage[utf8]{inputenc}
\usepackage{kurier}
\usepackage{amsmath}
\usepackage{amssymb}
\usepackage{xcolor}
\usepackage{epstopdf}
\usepackage{txfonts}
\usepackage{fancyhdr}
\usepackage{graphicx}
\usepackage{prettyref}
\usepackage{hyperref}
\usepackage{eurosym}
\usepackage{setspace}
\usepackage{units}
\usepackage{eso-pic,graphicx}
\usepackage{icomma}

\definecolor{darkblue}{rgb}{0,0,.5}
\hypersetup{pdftex=true, colorlinks=true, breaklinks=false, linkcolor=black, menucolor=black, pagecolor=black, urlcolor=darkblue}



\setlength{\columnsep}{2cm}


\newcommand{\arcsinh}{\mathrm{arcsinh}}
\newcommand{\asinh}{\mathrm{arcsinh}}
\newcommand{\ergebnis}{\textcolor{red}{\mathrm{Ergebnis}}}
\newcommand{\fehlt}{\textcolor{red}{Hier fehlen noch Inhalte.}}
\newcommand{\betanotice}{\textcolor{red}{Diese Aufgaben sind noch nicht in der Übung kontrolliert worden. Es sind lediglich meine Überlegungen und Lösungsansätze zu den Aufgaben. Es können Fehler enthalten sein!!! Das Dokument wird fortwährend aktualisiert und erst wenn das \textcolor{black}{beta} aus dem Dateinamen verschwindet ist es endgültig.}}
\newcommand{\half}{\frac{1}{2}}
\renewcommand{\d}{\, \mathrm d}
\newcommand{\punkte}{\textcolor{white}{xxxxx}}
\newcommand{\p}{\, \partial}
\newcommand{\dd}[1]{\item[#1] \hfill \\}

\renewcommand{\familydefault}{\sfdefault}
\renewcommand\thesection{}
\renewcommand\thesubsection{}
\renewcommand\thesubsubsection{}


\newcommand{\themodul}{Messtechnik Praktikum: Versuch Filter}
\newcommand{\thetutor}{Prof. Schöning}

\pagestyle{fancy}
\fancyhead[L]{\footnotesize{C. Hansen, D. Heffels, P. Motzfeld}}
\chead{\thepage}
\rhead{}
\lfoot{}
\cfoot{}
\rfoot{}

\title{\themodul{}, \thetutor}


\author{Christoph Hansen \and Dennis Heffels \and Pascal Motzfeld }

\date{}



\begin{document}

\maketitle

%Dieser Text ist unter dieser \href{http://creativecommons.org/licenses/by/4.0/}{Creative Commons} Lizenz veröffentlicht.


\tableofcontents

\newpage



\section{Versuch Filter 1}


\subsection{Beantwortung der Fragen}

\subsubsection*{Das Antwortsignal (Dämpfungsfaktor 2,5)}

Das Antwortsignal ist eine abschwellende Schwingung , das bei 4475 seinen ersten Peak bei ca 1,76 hat. Der erste Tiefpunkt befindet sich bei ca 4512 mit einer Höhe von ca 1,1, im weiteren Verlauf pendelt sich das Signal ab 4750 ziemlich gut bei einem Wert von ca 1,3 ein.


\subsubsection*{Das Antwortsignal (Dämpfungsfaktor 8)}

Bei einem Dämpfungsfaktor von 8 ist die Schwingung des Antwortsignals kaum noch zu erkennen, zudem ist der Zeitraum nachdem ein konstantes Signal erreicht wird, mit knapp 80 Skalenanteilen sehr viel kürzer als bei einer Dämpfung von 2,5. Außerdem ist der Beharrungswert hier kleiner als das Eingangssignal. Diese Änderung führen wir auf den gesteigerten Dämpfungsfaktor zurück, der dafür sorgt, das die Anzahl der Schwingungen um den Beharrungswert zurückgeht und das Eingangssignal zusätzlich vermindert.


\subsection{Eingangssignale}

In der Messtechnik wir zwischen verschiedenen Eingangssignale unterschieden, diese sind:

\begin{itemize}
\item[1)] deterministisch auch Sprungsignale genannt
\item[2)] stochastisch
\item[3)] periodisch z.B. ein Sinus
\item[4)] aperiodisch
\item[5)] stationär
\end{itemize}

Eine kleine Veranschaulichung wie Eingangs- und Ausgangssignal aussehen können liefert dieses Bild:

\begin{figure}[h]
\centering
\includegraphics[scale=0.9]{Eingangssignale.png}
\end{figure}


\subsection{Übertragungselement}

In einem Übertragungselement findet die Umwandlung von Messdaten statt. Eine Auswertung beinhaltet oft mehrere dieser Schritte, bis man das gewünschte Ergebnis erhält.

\begin{figure}[h]
\centering
\includegraphics[scale=0.7]{Uebertragungselement.png}
\end{figure}




\subsection{Auswertung der Ausdrucke}


\subsubsection*{Dämpfung 2,5}

Bei diesen Ausdruck entspricht $\unit[10]{mm} = \unit[25]{Skalenteilen}$, zusammen mit der Umrechnungsformel von Skalenteilen in $ms$ kommen wir auf folgende Formel:

\begin{align*}
\unit[T_x]{[ms]} &= \frac{ \frac{10}{25} \cdot \Delta \text{mm} \cdot 2}{5}
\end{align*}

Damit berechnen wir nun alle geforderten Zeiten:

\begin{align*}
&\text{Verzugszeit}:  &&T_u = \frac{ \frac{10}{25} \cdot \unit[2]{mm} \cdot 2}{5} = \unit[0,32]{ms} \\
&\text{Ausgleichszeit}:  &&T_g = \frac{ \frac{10}{25} \cdot \unit[7]{mm} \cdot 2}{5} = \unit[1,12]{ms} \\
&\text{Einschwingzeit}:  &&T_e = \frac{ \frac{10}{25} \cdot \unit[39]{mm} \cdot 2}{5} = \unit[6,24]{ms} \\
&\text{Anschwingzeit}:  &&T_a = \frac{ \frac{10}{25} \cdot \unit[8]{mm} \cdot 2}{5} = \unit[1,28]{ms} \\
&\text{Halbwertszeit}:  &&T_h = \frac{ \frac{10}{25} \cdot \unit[6]{mm} \cdot 2}{5} = \unit[0,96]{ms} \\
&\text{Überschwingweite (abgelesen)}:  &&v_m = 1,77 - 1,36 = \unit[0,41]{V} \\
&\text{Übertragungsbeiwert}:  &&K = \frac{\text{Beharrungswert}}{\text{Eingangssignal}} = \frac{\unit[1,36]{V}}{\unit[1]{V}} = \unit[1,36]{ } \\
\end{align*}


\newpage

\subsubsection*{Dämpfung 8}


Bei diesen Ausdruck entspricht $\unit[16,5]{mm} = \unit[10]{Skalenteilen}$, zusammen mit der Umrechnungsformel von Skalenteilen in $ms$ kommen wir auf folgende Formel:

\begin{align*}
\unit[T_x]{[ms]} &= \frac{ \frac{16,5}{10} \cdot \Delta \text{mm} \cdot 2}{5}
\end{align*}

Damit berechnen wir nun alle geforderten Zeiten:

\begin{align*}
&\text{Verzugszeit}:  &&T_u = \frac{ \frac{16,5}{10} \cdot \unit[1]{mm} \cdot 2}{5} = \unit[0,66]{ms} \\
&\text{Ausgleichszeit}:  &&T_g = \frac{ \frac{16,5}{10} \cdot \unit[42]{mm} \cdot 2}{5} = \unit[27,72]{ms} \\
&\text{Einschwingzeit}:  &&T_e = \text{Nicht aus Diagramm ablesbar,da Dämpfung zu groß} \\
&\text{Anschwingzeit}:  &&T_a = \frac{ \frac{16,5}{10} \cdot \unit[40]{mm} \cdot 2}{5} = \unit[26,4]{ms} \\
&\text{Halbwertszeit}:  &&T_h = \frac{ \frac{16,5}{10} \cdot \unit[28]{mm} \cdot 2}{5} = \unit[18,48]{ms} \\
&\text{Überschwingweite (abgelesen)}:  &&v_m = 0,77 - 0,79 = \unit[0,02]{V} \\
&\text{Übertragungsbeiwert}:  &&K = \frac{\text{Beharrungswert}}{\text{Eingangssignal}} = \frac{\unit[0,77]{V}}{\unit[1]{V}} = \unit[0,77]{ } \\
\end{align*}

\hfill \\

Nun eine Übersicht über die gewonnen Werte:

\hfill \\

\begin{tabular}{c|c|c}
 & Dämpfung 2,5 & Dämpfung 8 \\ 
\hline 
Verzugszeit [ms] & 0,32 & 0,66 \\ 
\hline 
Ausgleichszeit [ms] & 1,12 & 27,72 \\ 
\hline 
Einschwingzeit [ms] & 6,24 & nicht ablesbar \\ 
\hline 
Anschwingzeit [ms] & 1,28 & 26,4 \\ 
\hline 
Halbwertszeit [ms] & 0,96 & 18,48 \\ 
\hline 
Überschwingweite [V] & 0,41 & 0,02 \\ 
\hline 
Übertragungsbeiwert & 1,36 & 0,77 \\ 
\end{tabular} 


\newpage



\section{Versuch Filter 2}


\subsection{Beantwortung der Fragen}


\subsubsection*{Veränderungen im Signal}

Wir erkennen, dass die Amplitude des Ausgangssignals zunächst wächst und ab $\unit[60]{Hz}$ wieder abfällt. Weiterhin steigt die Phasenverschiebung mit der Frequenzerhöhung an, bis diese am Schluss bei ca $\unit[170]{^\circ}$ liegt.


\subsubsection*{Maximale Verstärkung}

Bei uns liegt die maximale Verstärkung mit einer Amplitude von $0,36$ bei $\unit[60]{Hz}$ vor.



\subsection{Auswertung und Diagramme}



\begin{tabular}{r|r|r|r|r|r|r|r|r}
Frequenz & C1    & C2    & Delta C & A     & C1    & C2    & $\Delta$ Skala & $\Delta$ ms \\
\hline
5     & 0,8   & 0,54  & 0,26  & 0,13  & 1864  & 1353  & 511   & 204,68 \\
10    & 0,8   & 0,54  & 0,26  & 0,13  & 1166  & 914   & 252   & 100,8 \\
20    & 0,81  & 0,53  & 0,28  & 0,14  & 810   & 683   & 127   & 50,8 \\
40    & 0,86  & 0,48  & 0,38  & 0,19  & 719   & 655   & 64    & 25,6 \\
60    & 1,03  & 0,31  & 0,72  & 0,36  & 658   & 616   & 42    & 16,8 \\
80    & 0,92  & 0,41  & 0,51  & 0,25  & 609   & 578   & 31    & 12,4 \\
100   & 0,56  & 0,77  & 0,21  & 0,105 & 552   & 577   & 25    & 10 \\
120   & 0,73  & 0,61  & 0,12  & 0,06  & 534   & 513   & 21    & 8,4 \\
150   & 0,7   & 0,64  & 0,06  & 0,03  & 505   & 488   & 17    & 6,8 \\
\end{tabular}%

\hfill \\
\hfill \\


\begin{tabular}{r|r|r|r|r|r|r|r|r}
C1    & C2    & $\Delta$ Skala & $\Delta$ ms & Phi   & G     & $\omega$ & Im    & Re \\
\hline
1349  & 1349  & 0     & 0     & 0     & 1,3   & 31,41 & 0     & 1,3 \\
914   & 914   & 0     & 0     & 0     & 1,3   & 62,83 & 0     & 1,3 \\
683   & 683   & 0     & 0     & 0     & 1,4   & 125,66 & 0     & 1,4 \\
656   & 654   & -2    & 0,8   & -11,25 & 1,9   & 251,32 & 1,832 & 0,47 \\
610   & 616   & -6    & 2,4   & -51,43 & 3,6   & 376,99 & -3,30 & 1,42 \\
566   & 571   & -11   & 4,4   & -127,74 & 2,5   & 502,65 & -2,18 & -1,21 \\
552   & 542   & -10   & 4     & -144  & 1,03  & 628,31 & 0,5 & 0,89 \\
504   & 513   & -9    & 3,6   & -154,28 & 0,6   & 753,98 & 0,20 & -0,56 \\
480   & 488   & -8    & 3,2   & -169,4 & 0,3   & 942,47 & 0,073 & 0,29 \\
\end{tabular}%



\newpage


\begin{figure}[h]
\centering
\includegraphics[scale=0.45]{Bode1.png}
\end{figure}

\hfill \\
\hfill \\


\begin{figure}[h]
\centering
\includegraphics[scale=0.45]{Bode2.png}
\end{figure}


\newpage

\begin{figure}[h]
\centering
\includegraphics[scale=0.45]{Nyquist.png}
\end{figure}


Das Nyquistdiagramm ist nicht aussagekräftig, da die Punkte nicht auf einem Halbkreis liegen. Wir können uns dies nur mit ungenügenden Messdaten erklären.


\end{document}
