\input{header.tex}


\begin{document}

\maketitle

%Dieser Text ist unter dieser \href{http://creativecommons.org/licenses/by/4.0/}{Creative Commons} Lizenz veröffentlicht.


\tableofcontents

\newpage



\section{Versuch Filter 1}


\subsection{Beantwortung der Fragen}

\subsubsection*{Das Antwortsignal (Dämpfungsfaktor 2,5)}

Das Antwortsignal ist eine abschwellende Schwingung , das bei 4475 seinen ersten Peak bei ca 1,76 hat. Der erste Tiefpunkt befindet sich bei ca 4512 mit einer Höhe von ca 1,1, im weiteren Verlauf pendelt sich das Signal ab 4750 ziemlich gut bei einem Wert von ca 1,3 ein.


\subsubsection*{Das Antwortsignal (Dämpfungsfaktor 8)}

Bei einem Dämpfungsfaktor von 8 ist die Schwingung des Antwortsignals kaum noch zu erkennen, zudem ist der Zeitraum nachdem ein konstantes Signal erreicht wird, mit knapp 80 Skalenanteilen sehr viel kürzer als bei einer Dämpfung von 2,5. Diese Änderung führen wir auf den gesteigerten Dämpfungsfaktor zurück.


\subsection{Eingangssignale}

In der Messtechnik wir zwischen verschiedenen Eingangssignale unterschieden, diese sind:

\begin{itemize}
\item[1)] deterministisch auch Sprungsignale genannt
\item[2)] stochastisch
\item[3)] periodisch z.B. ein Sinus
\item[4)] aperiodisch
\item[5)] stationär
\end{itemize}

Eine kleine Veranschaulichung wie Eingangs- und Ausgangssignal aussehen können liefert dieses Bild:

\begin{figure}[h]
\centering
\includegraphics[scale=0.9]{Eingangssignale.png}
\end{figure}


\subsubsection*{Dämpfung 2,5}

Bei diesen Ausdruck entspricht $\unit[10]{mm} = \unit[25]{Skalenteilen}$, zusammen mit der Umrechnungsformel von Skalenteilen in $ms$ kommen wir auf folgende Formel:

\begin{align*}
\unit[T_x]{[ms]} &= \frac{ \frac{10}{25} \cdot \Delta \text{mm} \cdot 2}{5}
\end{align*}

Damit berechnen wir nun alle geforderten Zeiten:

\begin{align*}
&\text{Verzugszeit}:  &&T_u = \frac{ \frac{10}{25} \cdot \unit[2]{mm} \cdot 2}{5} = \unit[0,32]{ms} \\
&\text{Ausgleichszeit}:  &&T_g = \frac{ \frac{10}{25} \cdot \unit[7]{mm} \cdot 2}{5} = \unit[1,12]{ms} \\
&\text{Einschwingzeit}:  &&T_e = \frac{ \frac{10}{25} \cdot \unit[39]{mm} \cdot 2}{5} = \unit[6,24]{ms} \\
&\text{Anschwingzeit}:  &&T_a = \frac{ \frac{10}{25} \cdot \unit[8]{mm} \cdot 2}{5} = \unit[1,28]{ms} \\
&\text{Halbwertszeit}:  &&T_h = \frac{ \frac{10}{25} \cdot \unit[6]{mm} \cdot 2}{5} = \unit[0,96]{ms} \\
&\text{Überschwingweite (abgelesen)}:  &&v_m = 1,77 - 1,36 = \unit[0,41]{V} \\
&\text{Übertragungsbeiwert}:  &&K = \frac{\text{Beharrungswert}}{\text{Eingangssignal}} = \frac{\unit[1,36]{V}}{\unit[1]{V}} = \unit[1,36]{ } \\
\end{align*}
\section{Versuch Filter 2}



\end{document}
